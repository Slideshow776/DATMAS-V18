In this chapter the different datasets used will be introduced. The goal of this project is to use as many datasets possible and then later evaluate them according to relevant results.

\section*{3.1 Folkehelseinstituttet}
The Institute of Public Health or Folkehelseinstituttet (fhi) have weekly updates\cite{fhi} on the development of the seasonal cases of influenza. The reports include numbers of diagnoses and graphs of flue-like illnesses. No numbers are appended to the flue-like illnesses and therefore this project will not take that into account, only diagnosed instances of influenza. Exact numbers are only included in the three last years, therefore the project only uses the seasons of the years 2015/2016, 2016/2017 and 2017/2018. The reports covers how many Norwegians seek treatment for influenza like illnesses and what kind of influenza viruses are circulating in the country, vaccine status and recommendations, as well as the overall prognosis of this season. The surveillance of influenza is based on doctors reporting flue-like illnesses of patients and testing those hospitalized for influenza viruses. Doctors report flue-like illnesses based on these symptoms: muscle pain, coughing, fever and the feeling of being sick.

\section*{3.2 Vegvesenet}
The Norwegian Public Roads Administration (NPRA), or Vegvesenet as it is called in Norwegian, have several different collections of data available for a number of different purposes. The motivation of this project requires traffic data of how many cars pass a certain registration station at a given time at a given position, the hypothesis for this that when people are ill they commute less and thus this shows on statistical data. Freely on their website \cite{vegvesenet} there are a few interesting options for this project. They have traffic information in a DATEX API, statistics in XLM and traffic index data relevant to the years before. It is important that the data collected is on a weekly basis atleast in order to compare it to the influenza data. The data on their website does not suffice for this purpose, traffic data is only registered on a yearly and monthly basis. Luckily upon further investigation and help from the NPRA better data was granted upon request, hidden from that available on their website. The data given contained a set of traffic registration stations throughout Norway. With this statistics of the daily traffic amount and spatial bounds can be derived showing the possible correlation influenza can have on traffic.

\section*{3.3 Twitter}
The reason twitter data is interesting is that it contains self reported instances of influenza before the patient or even if the patient visits a doctor for diagnosis and treatment. The pros are instant notification about possible influenza like illness and its spread against the cons of it being self reported. Twitter have several APIs available for public use, the one used in this project is the REST or search API which allows for searching against a set of keywords. The REST API is limited though, data accessible is only max 10 days old and the search limit is on a max of one hundred messages called 'tweets'. The other API of interest is the stream API which continually gets the latest tweets. In order to only get Norwegian tweets a set of geographical locations needs to be defined. The reason the stream API was not used is firstly because it requires a computer running on the internet continuously in order to get all the tweets. Secondly the data collected could become large slowing down other post-processing algorithms and taking up unnecessary storage. Lastly the stream API only provides a small set of the actual tweets tweeted, this means when searching for a specific term using the stream API some relevant tweets could go unnoticed and thus a search API is more appropriate for this task.