In this chapter, the different datasets used will be introduced. The goal of this thesis is to use as many datasets possible and then later evaluate them according to relevant results.

\section{The Norwegian Institute of Public Health}
The Norwegian Institute of Public Health (NIPH) have weekly updates\cite{fhi} on the development of the current influenza season as well as previous ones. The reports include numbers of diagnoses from general practitioners (GPs) considering influenza-like illness (ILI) and hospitalized virus observations. These are the main focus and acts as a baseline for other datasets to compare against. The virus observation numbers are included in the report, ILI symptoms are not, they are however both included in graphs. Upon further request, the ILI data was provided for the season of 2016/2017, and for the cities of Oslo and Bergen of the season of 2015/2016, 2016/2017 and 2017/2018. Exact numbers of the virus observations are only included for the three last years, therefore this thesis only uses the seasons of the years 2015/2016, 2016/2017 and 2017/2018. The reports also cover what kind of influenza viruses are circulating in the country and where, vaccine status and recommendations, as well as the overall prognosis of the current season. GPs report ILI based on these characteristics: muscle pain, coughing, fever and the feeling of being sick. The ILI numbers are perhaps of more interest since they are more accessible than virus observations that only counts for hospitalization. These two datasets provide the measurement basis other datasets are held up against.

\section{The Norwegian Public Roads Administration}
The Norwegian Public Roads Administration (NPRA) have several different collections of data available for a number of different purposes \cite{vegvesenet}. The main motivation for traffic data in this thesis is the hypothesis that when people are ill they commute less and thus this shows when surveying statistical details. Freely on their website \cite{vegvesenet} there are a few interesting options. They have traffic information in the standard traffic management exchange data structure (DATEX), application programming interfaces (API), statistics in an extensible markup language (XML) and traffic index data relevant to the years before. It is important for this thesis that the data collected is on a weekly basis at least in order to compare it to the influenza data. It turned out that the data on their website did not suffice for this purpose, they only had a temporal resolution of months or years while this thesis needs a temporal resolution of weeks or better. The data given contained a set of traffic registration stations throughout Norway. Data provided was on a weekly basis and also on an hourly basis for a subset of the original traffic registration stations provided. With this statistics of the traffic amount and spatial bounds can be derived showing the possible correlation influenza can have on commuting traffic. The regions of interest are the whole of Norway and the three cities of Stavanger, Bergen, and Oslo.

\section{Twitter}
The reason twitter data is interesting is that it contains self-reported instances of influenza on an individual level. These self-reported cases may even occur without the patient visiting a doctor, and so capture otherwise non-reported instances of ILI. The advantages are an instant notification about possible ILI and its spread, against the disadvantages of it being self-reported and thus somewhat unreliable. Twitter has several APIs available for public use, the one used in this project is the representational state transfer (REST) API or 'search API' which allows for searching against a set of keywords. The REST API is limited though, data access is roughly only maximum 10 days old and the search limit is on a maximum of one hundred messages called 'tweets'. The other API of interest is the stream API which continually gets the latest tweets. In order to only get Norwegian tweets, a set of geographical locations needs to be defined. The reason the stream API was not used is firstly that it requires a computer running on the internet continuously in order to get all the desired tweets. Secondly, the data collected could become large slowing down other post-processing algorithms and taking up unnecessary storage. Lastly, the stream API only provides a small set of the actual tweets tweeted, this means when searching for a specific term using the stream API some relevant tweets could go unnoticed and thus a search API is more appropriate for this task.

\section{Kolumbus}
Kolumbus is the public transportation administration in the state of Rogaland in Norway, this includes Stavanger, a city of interest. Unfortunately, Kolumbus provides no API, but on further request data of monthly passenger travel was provided from the years of 2015-2017.

\section{Ruter}
Ruter is the public transportation administration in the state of Oslo in Norway. Unfortunately, Ruter's API does not include passenger or tickets sold information, this was however provided on request for the years 2015, 2016, 2017 and up till 27 of February for the year 2018 on a daily basis.