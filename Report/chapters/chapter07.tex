This thesis presents a program that is a prototype of what an automated collection of multi-source spatial information for emergency management systems may look like. It contains a basic assembly of what such a system would be embodied of and is also modular and scalable making future work possible by integrating new sources from other agencies and further development of structure and accessory features.
\\
\\
The program visualises collected data both spatially on a map and by traditional graphs. The information is presented by category and with multiple tools to help with investigative analysis. Tools like moving, zooming in and out on graph assets, capturing graph state, adjusting subplot parameters, comparison of graphs, querying graphs and map visualisation. A considerable amount of time was spent creating this program.
\\
The early months of this thesis's time scope were characterised by collecting data from various agencies to be used in the backend. This was notably a tough grind as initiating contact could take several weeks followed by multiple rounds of communication back and forth, with reminders and careful explanations of what was intended and needed. The cooperation with the involved agents varied in quality, but the overall patronage was sufficient. This process involved some waiting days with modest development, this is a common occurrence in professional work.
\\


Automated collection of multi-source spatial information for emergency management such as creating a responsive real-time reactionary system for influenza is feasible.
The automation part may not become practical for years to come as the general Norwegian public API infrastructure is notwithstanding at the current time. However collecting data manually is still a reasonable effort. Implementing more relevant sources should also be a priority, as well as collecting spatial data for specific regions of Norway.
\\
This thesis shows that an automated collection of multi-source spatial information for emergency management is achievable, however, this would require much more resources than a single master student can offer on a six month time-scope. Further development of such a program, be that for influenza purposes or other and as long as the data is aggregated beyond citizen identification, would be highly ethical as it would be an exceptional aid to ameliorate the purposed predicament. Efforts to further support data collection of citizen behaviour on a macro scale should be initiated such as to encourage additional endeavours.
It is important that information procured from such big data analysis is closely tied to further research and the NIPH's available viral, ILI, and other data they may have in order to make reliable assumptions and thus, initiatives.\\
It seems that there is a potential untapped market with the dealing of retrieving multi-source data for the purposes of real-time responsive system that derive information and exploits this.












































