This thesis presents a program that is a prototype of what an automated collection of multi-source spatial information for emergency management systems may look like. It contains a basic assembly of the system and describes the various components of the program. The program visualises collected data both spatially on a map and by traditional graphs. The information is presented by category and contains multiple tools to help with investigative analysis. Tools like moving, zooming in and out on graph assets, capturing graph state, adjusting subplot parameters, comparison of graphs, querying graphs and map visualisation. The system is modular and scalable and allows for the addition of further features, as well as an integration of new data sources from other agencies.\\

The early months of this thesis's time scope were characterized by collecting data from various agencies to be used in the backend. This was notably a tough grind, as getting access to data would take several weeks followed by multiple rounds of communication back and forth, with reminders and careful explanations along the way. The cooperation with the involved agents varied in quality, but the overall service was sufficient. This process involved some waiting days with modest development; this is a common occurrence in professional work.\\

Automated collection of multi-source spatial information is a critical component of a real-time societal indicator surveillance system, as such systems are dependent on access to large amounts of data from a variety of sources. The automation part may not become practical for years to come, as the general Norwegian public API infrastructure is not sufficiently implemented at the current point in time. Implementing more relevant sources should also be a priority, as well as collecting spatial data for specific regions of Norway. Big Data acquisition and analysis efforts must be coordinated with further research on method development, to integrate the methods into the next generation of flu surveillance systems. Combining information from multiple sources provides a powerful capability to assist in emergency management, and efforts to additionally support data collection of citizen behaviour on a macro scale should be encouraged.





