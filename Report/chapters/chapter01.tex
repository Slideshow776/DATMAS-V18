\section{Background}
The power to obtain enough information to detect possible trends of influenza seasons depends on successful integration between a multitude of different participants. Automatic extraction and processing of data is paramount for efficient analysis and gives a solid basis for an autonomous pathological detection system. Scalability is important in merging new relevant datasets as they become available in an ever-growing societal infrastructure. This proposed technology would become an influential part of a bigger foundation intertwined with a robust knowledgeable and organizational means to mobilize assets in order to respond to possible outbreaks as or even before they start.
\newline \\
Influenza is an exceedingly contagious viral infection which gives high fever, general pain, and respiratory symptoms. An estimated five to ten percent of the population becomes infected during a yearly winter season.\\ The virus is especially dangerous to the elderly and to pregnant people from the second-trimester \cite{fhi_sykdommer}.


\section{Objectives}
This paper describes a plausible examination of the viability of monitoring, collecting and analyzing obtainable relevant data for a self-sufficient influenza seasonal recognition system. The management of seasonal influenza outbreaks is handled by public health officials and epidemiologists with the use of the national surveillance system provided by the Norwegian Institute of Public Health (NIPH)\cite{niph}. The Norwegian Syndromic Surveillance System (NorSySS) collects influenza-like illnesses (ILI) from general practitioners (GPs)\cite{NorSySS}. These provide the means to monitor current influenza seasons with delay and as a basis to survey urban real-time datasets. The main thesis of this project is as influenza develops this reveals subtle patterns in societal behaviour that is detectable through a variety of mediums, e.g urban datasets from sewage, public transportation, medicinal purchases, recreational habits, social media and other such sources of public information.

\section{Outline}
The thesis is structured into seven chapters.
\newline \\Chapter 2 describes related works 
\newline \\Chapter 3 mark out in detail the datasets used by this project, describes and give explanation to relevance, challenges, limitation and rewards.
\newline \\Chapter 4 outlines the implementation and graphical results of the datasets used in chapter 3.
\newline \\Chapter 5 shows the results.
\newline \\Chapter 6 discusses the results.
\newline \\Chapter 7 concludes the thesis, discusses constraints and possible future work as well as other suggestions.