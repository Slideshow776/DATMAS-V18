\thispagestyle{plain}
\begin{center}
	\Large
	\textbf{Automated collection of multi-source spatial information for emergency management}
	
	\vspace{0.4cm}
	\large
	Tracking the influenza seasons
	
	\vspace{0.4cm}
	\textbf{Sandra Moen}
	
	\vspace{0.9cm}
	\textbf{Abstract}
\end{center}
Yearly influenza epidemics carries a tremendous societal cost and leads to a large loss of life and an immense strain on the national health care systems. People that become sick are less productive and the overall well-being is drastically reduced for the duration of the individuals period of illness as well as the community during a flu season. Efficient flu control and management efforts directly save lives and resources.\\

Combating the impact of these outbreaks is dependent on a situational understanding of societal factors, to understand the impact and severity of the ongoing outbreak. This project aims to explore the possibilities to detect influenza outbreaks as soon as they are happening by examining relevant datasets available. Information about different aspects of people's lives on a grand scale reveals patterns and trends that could be linked to an epidemic, and thus provide a behavioural-based detection mechanism.\\

The results show that creating an automated responsive real-time warning system to reveal influenza emergence is achievable, and the program presented by this thesis is a prototype of what such a warning system may look like. It contains a basic assembly of core features and is also modular and scalable making future work possible. Developing such a program would augment the current clinical-based surveillance systems with one that monitors societal indicators for potential outbreaks, and provide an excellent benefit to society ameliorating the purposed predicament.





