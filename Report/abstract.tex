\thispagestyle{plain}
\begin{center}
	\Large
	\textbf{Automated collection of multi-source spatial information for emergency management}
	
	\vspace{0.4cm}
	\large
	Tracking the influenza seasons
	
	\vspace{0.4cm}
	\textbf{Sandra Moen}
	
	\vspace{0.9cm}
	\textbf{Abstract}
\end{center}
Influenza epidemics cost both lives and a tremendous amount of resources for any country. People that become sick are less productive and the overall well-being is drastically reduced for the duration of the individuals period of illness as well as the community during a flu season. The ability to reduce the spread of infectious diseases saves both lives and resources as well as it is an improvement of the quality of life. \\

This project aims to explore the possibilities to detect influenza outbreaks as soon as they are happening by examining relevant datasets available. Information about different aspects of people's lives on a grand scale reveals patterns and trends that could be linked to an epidemic, and thus prove useful for active measurements against further spread on an early début. \\

The results show that creating an automated responsive real-time warning system to reveal influenza emergence is achievable, and the program presented by this thesis is a prototype of what such a warning system may look like. It contains a basic assembly of core features and is also modular and scalable making future work possible. Developing such a program, be that for influenza purposes or other, would provide an excellent benefit to society to ameliorate the purposed predicament.