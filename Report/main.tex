\documentclass[12pt, twoside]{report}

\usepackage[utf8]{inputenc}
\usepackage{graphicx}
\graphicspath{{images/}}
\usepackage[a4paper,width=150mm,top=25mm,bottom=25mm,bindingoffset=6mm]{geometry}
\usepackage{fancyhdr}
\pagestyle{fancy}

%\usepackage[style=alphabetic, sorting=none]{biblatex}
%\addbibresource{references.bib}

\usepackage{pdfpages}
\usepackage{array}
\usepackage[section]{placeins} % TODO: denne tvinger bilder på plass, kanskje ta vekk?
\usepackage{hyperref}
\hypersetup{
    colorlinks=true, %set true if you want colored links
    linktoc=all,     %set to all if you want both sections and subsections linked
    linkcolor=black ,  %choose some color if you want links to stand out
    citecolor=[rgb]{0.3,0.0,1.0},
}
\usepackage{url}


\title{Automated collection of multi-source spatial information for emergency management}
\author{Sandra Moen}
\date{Spring 2018}
\begin{document}

\includepdf[pages=-]{englishFrontPageUiS_WIP_2.pdf}

\begin{titlepage}
	\begin{center}
		\vspace*{1cm}
		
		\Huge
		\textbf{Automated collection of multi-source spatial information for emergency management}
		
		\vspace{0.5cm}
		\LARGE
		Tracking the influenza seasons
		
		\vspace{1.5cm}
		
		\textbf{Sandra Moen}
		
		\vfill
		
		A thesis presented for the degree of \\
		Master of Science in Computer Science
		
		\vspace{0.8cm}
		
		\includegraphics[width=0.4\textwidth]{university}
		
		\vspace{0.8cm}
		
		\LARGE
		Department of Electrical Engineering and Computer Science\\
		University of Stavanger\\
		Norway\\
		Spring 2018
		
	\end{center}
\end{titlepage}

\thispagestyle{plain}
\begin{center}
	\Large
	\textbf{Automated collection of multi-source spatial information for emergency management}
	
	\vspace{0.4cm}
	\large
	Tracking the influenza seasons
	
	\vspace{0.4cm}
	\textbf{Sandra Moen}
	
	\vspace{0.9cm}
	\textbf{Abstract}
\end{center}
Yearly influenza epidemics carries a tremendous societal cost and leads to a large loss of life and an immense strain on the national health care systems. People that become sick are less productive and the overall well-being is drastically reduced for the duration of the individuals period of illness as well as the community during a flu season. Efficient flu control and management efforts directly save lives and resources.\\

Combating the impact of these outbreaks is dependent on a situational understanding of societal factors, to understand the impact and severity of the ongoing outbreak. This project aims to explore the possibilities to detect influenza outbreaks as soon as they are happening by examining relevant datasets available. Information about different aspects of people's lives on a grand scale reveals patterns and trends that could be linked to an epidemic, and thus provide a behavioural-based detection mechanism.\\

The results show that creating an automated responsive real-time warning system to reveal influenza emergence is achievable, and the program presented by this thesis is a prototype of what such a warning system may look like. It contains a basic assembly of core features and is also modular and scalable making future work possible. Developing such a program would augment the current clinical-based surveillance systems with one that monitors societal indicators for potential outbreaks, and provide an excellent benefit to society ameliorating the purposed predicament.







\chapter*{Acknowledgements}
This thesis is considered an impressive achievement for the author, it was completed in spite of hardships endured. Under no circumstance should this thesis be considered a Norwegian accomplishment, for the oppression suffered they are deemed unworthy.
\newline \\
This thesis was written for the Department of Electrical Engineering and Computer Science at the University of Stavanger. Creating a means to solve problems that limit peoples lives have always been a real motivator. Predicting the flu season and hindering it in early stages would save an enormous amount of resources and improve life quality, this would be very rewarding.\\
A special thanks to the supervisor for this project from the University of Stavanger Professor Erlend Tøssebro for his enthusiastic guidance and involvement, and the initiator who inspired incentive to the creation of this project as well as his continuous helpful guidance and involvement research Fellow Lars Ole Grottenberg.

\setcounter{secnumdepth}{5}
\setcounter{tocdepth}{5}
\tableofcontents
\listoffigures
\listoftables

\chapter{Introduction}
\section{Background}
Influenza is a highly contagious viral infection which gives high fever, general pain and respiratory symptoms. An estimated five to ten percent of the population becomes infected during a yearly winter season.\\ The virus is especially dangerous to the elderly and to pregnant people from the second trimester.
% \cite https://fhi.no/nettpub/smittevernveilederen/sykdommer-a-a/influensa/


\section{Objectives}

\section{Structure}
The thesis is structured into ... chapters.\\
Chapter 1, ...\\
Chapter 2, ...\\
Chapter 3, ...\\
Chapter 4, ...\\
Chapter 5, ...\\
Chapter 6, ...\\
Chapter 7, ...\\

\chapter{Related Works}
This section looks at previous work in similar fields. It starts with presenting the paper that offer the idea that this thesis further explores, and then looks at past research on using Twitter and critical infrastructure data for similar tasks.

\section{Spatiotemporal information from urban systems}
In the novel study of "Detecting flu outbreaks based on spatiotemporal information from an urban system", which is the base idea for this thesis, Grottenberg et al. \cite{spatiotemp_urban_sys} outlines a design for a system for surveillance of flu outbreaks. Emphasis on the belief that real-time data flows could prove useful in both understanding social functions during disasters and crisis as well as give " ... actionable intelligence for use in influenza management efforts.". The goal would be to extend the already implemented infrastructure with an approach to monitor human behaviour in trends throughout the influenza activity in hope for discrepancies detected through spatial analysis on important measurements. The borrowed figure \ref{fig:grottenberg} from his article sums up what this thesis hopes to accomplish, namely to find a correlation between different datasets and the datasets from the Norwegian public health institution (NIPH), this interference of public behaviour would become visible in essential criterion.
This short read \cite{spatiotemp_urban_sys} is recommended as it gives a more in-depth understanding of the incentive for this thesis.

\begin{figure}[h]
\includegraphics[width=16cm]{grottenberg}
\centering
\caption{Figure from Grottenberg et al. \cite{spatiotemp_urban_sys}}
\label{fig:grottenberg}
\end{figure}


\section{Spatiotemporal information from VGI}
Volunteered Geographic Information (VGI) is peer-produced crowd spatial data for use in crisis responses. Mobilizing digital volunteers to help with disastrous events alleviates the data needed by relief agents, VGI is peer-produced spatial data that is highly up-to-date. In 2010 the Haiti Earthquake levelled many official government buildings and with them access to official mapping resources\cite{palen2015success}. In just a few days volunteers contributed to OpenStreetMap\cite{OpenStreetMap} (OSM) and created an even better map of Haiti using satellite images by individually identifying map resources. A similar approach was initiated during the 2015 Nepal earthquake\cite{hu2016task}. Anderson et al\cite{anderson2018crowd} describes methods for evaluating the quality of VGI and to the development of "... rapid metrics of quality for digital data generated under socially distributed conditions ...". They reason that peer production platforms will be a more integrated part of disaster management and that when the risk of lives and infrastructure is present a solid basis for quality control of VGI information should be established whenever VGI is used.

\section{Data management and critical infrastructure}
This thesis touches upon data management and development of crisis response systems. The proposed system would act as a tool in a larger system in the development of support decision making in the event of an epidemic influenza preparedness and outbreak. 

Responding to extensive crisis or disasters requires coordination between a multitude of relief agencies, and this demands the right information at the right time. A system that can detect an emergence of a possible influenza outbreak would be an aiding factor to this. Gonzales et al. \cite{gonzalez2009framework} goes into general details of how the quality of information during a crisis response is important and how to better coordinate relief agencies with the right information at the right time. Their report includes a case study where simulation of interagency crisis response by the port of Rotterdam in the Netherlands, in particular this case study emphasise the qualitative trial and strategy throughout interagency crisis management. Extensive emergencies requires involvement of a multitude of relief and other regional service assets to cooperate and share relevant and timely information. The specifics of the case study simulates the collision of a containership with a passenger ship where the containership explodes and leaks hazardous chemicals. Responding to such a catastrophic tense event requires cooperation of multiple authorities from professional representatives guided by regional and port experts. Gonzales et al. concludes that designing a computer based system for management and automation services of a work flow information conductor would better the over all quality of response and guidance. The system proposed by this thesis could be a module of such a system. 

Machine-learning algorithms may also be of use in spatiotemporal analysis of social media data for disasters and damage assessment. Resch et al \cite{resch2018combining} explains how the current management of disasters have several shortcomings that can be solved by machine-learning topic models and spatiotemporal analysis. Temporal lags and limited resolution of information prevents successful and accurate resource deployment, advantages of new approaches with real-time collecting of data, like social media and other crowdsourcing networks "can significantly improve disaster management". Resch et al proposes a new approach to analyse social media with the combination of semantic machine-learning algorithms with spatio and temporal analysis. The challenge is detecting data flow continuously without prior analysis and knowledge about the event in question. Their results show remarkable improvement to accurate event tracking and other hotspots, disaster management and valuable insight to affected regions and assets.

Simulation modules could also be added to this system. This thesis is not a simulation tool but it is worth mentioning that there are several such proposed models of influenza and other disease simulation implementations. Shao et al. \cite{shao2016forecasting} ask the question of whether it is possible by monitoring public urban data by designing a social network sensor for epidemics to predict the coming outline of an overall epidemic, and simulates this. Developing sufficient heuristics in order to adpot social sensors to forecast influenza outbreaks when probabilistic views of structure of simulated influenza propagations interests public health dignitaries and govermental stratagem designers. There are many more simulation tools, another is proposed by Stein et al. \cite{stein2012development} which models an influenza outbreak in two provinces of Lao. Stein et al's framework proposes that planning for influenza outbreaks is an exigent engagement that requires predictive models to better evaluate responsive strategies. Stein et al freely offers their simulation too called AsianFluCap on their website and describes it as ".. a user-friendly, comprehensive and flexible simulation tool which can be used by decision makers involved in pandemic preparedness to estimate and compare the impact on health care resource capacity during different pandemic scenarios.". Simulations are a way of preparing and training in order to reveal flaws and evaluation of response plans and deployment of limited health care resources, and raise awareness of surges in sudden resource demands during pandemics, especially so where such resources are scarce and efficient delegation is important.

\section{The Ebola epidemic}
The west African Ebola viral haemorrhagic fever (VHF) epidemic lasted from 2013 to 2016 and spread to a wide part of the globe. Ebola causes fever, sore throat, muscular pain, headaches and lastly internal haemorrhage (internal bleeding), the death rate is about 25\% to 95\% with an average of 50\%\cite{who_ebola}. \\

Tom Koch\cite{koch2016ebola} with his international journal of epidemiology "Ebola in West Africa: lessons we may have learned" hopes that "... future disease outbreaks in rural areas with minimal resources can be better and more rapidly assessed.". Koch highlights the importance of ecological mapping to spatially identify environmental status that actively encourages disease opulence and expansion. Mapping the terrain and human assets with a geographical positioning system (GPS) provides practical means of ameliorating recurring pandemics. \\

An early response to emergency incidents is necessary for efficient containment, and mapping disease contributes to that objective. Koch further describes mapping as an important surveillance spatial tool to identify and contain outbreaks in his commentary "Mapping medical Disasters: Ebola Makes Old Lessons, New"\cite{koch2015mapping}. Knowledge about the location of disease and extent of official health resources provides more time to asses the situation and respond. The 2014 Ebola epidemic failed to survey the seriousness of the outbreak and dreadful events followed. Among the lessons learned from this happening is that the need for detailed medical mapping as soon as possible is paramount for a potential contagion. These technologies matter and are important to implement when resources are met and laid out for, collecting data to serve the public health as a warning system is something to be strived for.\\ 

During the Ebola disaster in 2014-2015 Médecins Sans Frontières\cite{gis_support} situated devoted Geographic Information Systems (GIS) officers to aid epidemiologists in the creation of topical maps to further support the operation. GIS was greatly beneficial to logistics, epidemiologists, and health promotion by providing knowledge about current disease hotspot flares and acting as a warning system for surrounding districts. VGI was also used\cite{moeller2015mapping} in mapathons (map creating marathons) on the initiative of the American Red Cross in cooperation with the Humanitarian OpenStreetMap Team.\\ 

With sufficient technological spatial data infrastructure, this process could be automated, and an even more effective emergency management system could be devised. The Ebola outbreak of 2013-2016 goes to show the severity of pandemics, and systems can be developed to effectively combat infectious outbreaks. Even though Ebola is a more serious illness than influenza it goes to show that emergency management systems is sorely needed and have a multitude of applications. Research, development and implementation in many situations overall betters quality and realisation, and the Ebola incident gives insights into managing influenza outbreaks by such systems even in Norway.




\section{Seasonal influenza}
TODO: tenk over hvordan et system for å advare mot en ny Spanskesyke kunne ha sett ut. Spanskesyken var en av de mest alvorlige influensaepidemiene i moderne tid.\\

Seasonal influenza, like Ebola, is a recurring disease and has the potential to spread worldwide and thus becoming a pandemic affliction. New undertakes to influenza prevention and treatment management both seasonal and pandemics are beneficial. In their article Catharine Paules and Kanta Subbarao \cite{article_Paules} describes the "... clinical presentation, transmission, diagnosis, management, and prevention of seasonal influenza infection.", and outlines that there are two forms of influenza outbreaks that occur globally: seasonal epidemics caused by type A and type B virus, and sporadic pandemics only caused by type A viruses. 






\section{Twitter}
"Twitter is an online news and social networking service on which users post and interact with messages known as "tweets"."\cite{twitter_twitter}. A number of studies have been performed on the information that the users of Twitter generate. These studies analyse millions of tweets to extract aggregate information. Researchers have studied tweets to reveal political opinions\cite{twitter_politic}, measure public health\cite{twitter_flu_trends}, linguistic sentiments\cite{twitter_linguistics} and even environmental phenomena such as earthquakes\cite{twitter_earthQuake}. Achrekar et al.\cite{twitter_flu_trends} examines tweet flu trends and compares them with actual influenza data. The results show a high correlation between self-reported instances of flu-like illnesses (ILI) and reported ILI by public health providers. Achrekar references claims that early prevention limits the spread of infectious diseases and that twitter data is an 'untapped data source' that actually is quite reliable. Another report by Byrd et al.\cite{byrd2016mining} also evidence how Twitter surveillance and classifying tweets by sentiment characterstics exactly identify users with ILI symptoms in selected cities in real-time, and also speculates usage of this technology in application of other disease protection systems. This demonstrates how social media can be  used to predict real-world consequences, and gives credibility to usage in this thesis. \\Michal J. Paul and Mark Dredze \cite{twitter_what_you_tweet} also conducted research on the usage of twitter data to measure population characteristics. In their conclusion twitter data from many users divulges reliable information about a certain topic of interest and in particular public health. They further discuss the pros and cons namely that self-reported is low cost and rapid transmission, whereas on the other side this is a 'blind authorship, lack of source citation and presentation of opinion as fact'. Certainly twitter messages may be false on an individual level, but however when taking into account thousands or even millions of messages this seems not plausible on a bigger scale. Albuquerque et al. \cite{de2015geographic} describes how they were able to extract useful information via twitter to better acquire information about a flood phenomena in German rivers, and combining this with authoritative data for disaster management. They write that social media messages gives a valuable and useful information to further aid and manage disasters as an addition to other sources , in a way this is practically the same as asking volunteers for help. For these reasons twitter data is used in this thesis as it proves an interesting and unique source of relevant information.







\chapter{Datasets used}
In this chapter the different datasets used will be introduced. The goal of this project is to use as many datasets possible and then later evaluate them according to relevant results.

\section*{3.1 Folkehelseinstituttet}
The Institute of Public Health or Folkehelseinstituttet (fhi) have weekly updates\cite{fhi} on the development on the current influenza season as well as previous ones. The reports include numbers of diagnoses from general practitioners (GPs) considering flue-like symptoms (FLS), and hospitalized virus observations with graphs of both. No numbers are appended to the FLS but upon further request this was provided. Exact numbers are only included for the three last years, therefore the project only uses the seasons of the years 2015/2016, 2016/2017 and 2017/2018. The reports covers how many Norwegians seek treatment for FLS and what kind of influenza viruses are circulating in the country and where, vaccine status and recommendations, as well as the overall prognosis of this season. GPs report flue-like symptoms based on these characteristics: muscle pain, coughing, fever and the feeling of being sick.

\section*{3.2 Vegvesenet}
The Norwegian Public Roads Administration (NPRA), or Vegvesenet as it is called in Norwegian, have several different collections of data available for a number of different purposes. The motivation of this project requires traffic data of how many cars pass a certain registration station at a given time at a given position, the hypothesis for this that when people are ill they commute less and thus this shows on statistical data. Freely on their website \cite{vegvesenet} there are a few interesting options. They have traffic information in a DATEX API, statistics in XLM and traffic index data relevant to the years before. It is important that the data collected is on a weekly basis atleast in order to compare it to the influenza data. The data on their website does not suffice for this purpose, traffic data is only registered on a yearly and monthly basis. Luckily upon further investigation and help from the NPRA better data was granted upon request, hidden from that available on their website. The data given contained a set of traffic registration stations throughout Norway. With this statistics of the daily traffic amount and spatial bounds can be derived showing the possible correlation influenza can have on traffic. The regions in interest is the whole of Norway and the three cities of Stavanger, Bergen and Oslo.

\section*{3.3 Twitter}
The reason twitter data is interesting is that it contains self reported instances of influenza before the patient or even if the patient visits a doctor for diagnosis and treatment. The advantages are instant notification about possible influenza like illness and its spread, against the disadvantages of it being self reported and thus somewhat unreliable. Twitter have several APIs available for public use, the one used in this project is the REST or search API which allows for searching against a set of keywords. The REST API is limited though, data accessible is roughly only maximum 10 days old and the search limit is on a maximum of one hundred messages called 'tweets'. The other API of interest is the stream API which continually gets the latest tweets. In order to only get Norwegian tweets a set of geographical locations needs to be defined. The reason the stream API was not used is firstly because it requires a computer running on the internet continuously in order to get all the desired tweets. Secondly the data collected could become large slowing down other post-processing algorithms and taking up unnecessary storage. Lastly the stream API only provides a small set of the actual tweets tweeted, this means when searching for a specific term using the stream API some relevant tweets could go unnoticed and thus a search API is more appropriate for this task.

\section*{3.4 Kolumbus}
Kolumbus is the public transportation administration in the state of Rogaland in Norway, this includes Stavanger, a city of interest. Unfortunately Kolumbus provides no API, but on further request data of monthly passenger travel was provided.

\chapter{Implementation}
This chapter describes how the use of the different datasets were implemented.

\section{Folkehelseinstituttet}
The data contained two different sets, and it was a simple job to plot them in a graph. Figure \ref{fig:infstat} show the three last seasons of influenza. The plotting was done manually as fhi only provides the data in pdf format.

\begin{figure}[ht]
\includegraphics[width=16cm]{influenza_15_till_18}
\centering
\caption{Influenza seasons}
\label{fig:infstat}
\end{figure}

Figure \ref{fig:ilsstat} shows the ILS of the year 2016/2017. This was not done manually as data was provided in a simple .xlsx file 

\begin{figure}[ht]
\includegraphics[width=16cm]{ILS_16_till_17}
\centering
\caption{Influenza like symptoms season 2016/2017}
\label{fig:ilsstat}
\end{figure}

\newpage

\section{Vegvesenet}
From the XML statistics some simple graphs were created in python showing the total annual traffic on Norwegian roads from 2002 to 2015 as seen in figure \ref{fig:anualtotal}. 

\begin{figure}[ht]
\includegraphics[width=16cm]{xml_02_15_annual_total}
\centering
\caption{Annual traffic 2002-2015}
\label{fig:anualtotal}
\end{figure}

Also derived from this the annual traffic of the two cities Bergen and Oslo, which are towns in interest. Figure \ref{fig:anualbergen} shows the traffic in Bergen, and figure \ref{fig:anualoslo} show the traffic in Oslo.

\begin{figure}[ht]
\includegraphics[width=16cm]{xml_02_15_annual_bergen}
\centering
\caption{Bergen traffic 2002-2015}
\label{fig:anualbergen}
\end{figure}

\begin{figure}[ht]
\includegraphics[width=16cm]{xml_02_15_annual_oslo}
\centering
\caption{Oslo traffic 2002-2015}
\label{fig:anualoslo}
\end{figure}
The dataset is an XLM file structure that is downloaded from the NPRA manually. A python program was created that reads through all rows and collects the relevant columns into an array and then draws a graph. For the annual graph every month of every year was collected. For the towns of Bergen and Oslo the correct roads were identified and loaded from a separate text file, then every year of every month of those roads were collected, loaded into an array and the drawn as a graph. The separate text file is to make it easy to edit should these roads change in the future.
The problem of using these datasets is that the data is an average calculation of monthly traffic, this is too coarse for comparison against the influenza data as they are on a weekly basis. Luckily upon further investigation and help from the NPRA better data was accessible upon request, hidden from that available on their website. A set of traffic registration stations was needed to define the temporal bounds of each area of interest. Defined are the towns of Oslo, Stavanger and Bergen, as well as the whole of Norway on a level 1 basis. The level 1 registrations ensures continually registration throughout the year, and is exactly what this project requires.

Figures \ref{fig:weeklybergen}, \ref{fig:weeklyoslo} and \ref{fig:weeklystavanger} shows the traffic on a weekly basis. This provides a better resolution for better analysis.
\begin{figure}[ht]
\includegraphics[width=16cm]{NPRA_13_17_weekly_bergen}
\centering
\caption{Weekly data of the city of Bergen}
\label{fig:weeklybergen}
\end{figure}

\begin{figure}[ht]
\includegraphics[width=16cm]{NPRA_13_17_weekly_oslo}
\centering
\caption{Weekly data of the city of Oslo}
\label{fig:weeklyoslo}
\end{figure}

\begin{figure}[ht]
\includegraphics[width=16cm]{NPRA_13_17_weekly_stavanger}
\centering
\caption{Weekly data of the city of Stavanger}
\label{fig:weeklystavanger}
\end{figure}

Figure \ref{fig:boundsbergen}, \ref{fig:boundsoslo} and \ref{fig:boundsstavanger} shows the different geospatial bounds used to define the cities. The green cirlces with numbers inside show where and how many traffic registration stations there are.

\begin{figure}[ht]
\includegraphics[width=16cm]{nivaa_1_bergen}
\centering
\caption{Geospatial bounds of Bergen}
\label{fig:boundsbergen}
\end{figure}

\begin{figure}[ht]
\includegraphics[width=16cm]{nivaa_1_oslo}
\centering
\caption{Geospatial bounds of Oslo}
\label{fig:boundsoslo}
\end{figure}

\begin{figure}[ht]
\includegraphics[width=16cm]{nivaa_1_stavanger}
\centering
\caption{Geospatial bounds of Stavanger}
\label{fig:boundsstavanger}
\end{figure}

\newpage\newpage

\section{Twitter}
Using the REST search API it was paramount that in order to build a sufficient dataset acquiring and collecting data had to begin as soon as possible in order to collect enough data for this project. A simple python program was created that takes the input of the API keys and the keywords to be searched upon . The program ensures that no duplicate messages are recorded, and the limit of a hundred tweets was overcome simply by searching for yet another hundred from the last date of the previous hundred, until the date limit was reached.
The output is appended to a file in this format: id, date, location, tweet.

A simple analysis tool for the twitter data was created by simply counting how many messages there are. The idea is that during influenza seasons numbers of tweets will rise and vice versa when off the season. Figure \ref{fig:twitterAnal} shows the results.

\begin{figure}[ht]
\includegraphics[width=16cm]{twitter_tweets_2018}
\centering
\caption{Tweets concerning ILS of 2018}
\label{fig:twitterAnal}
\end{figure}

\section{Kolumbus}
The data provided from Kolumbus was in a .png format and had to be converted. From there it was a simple job to plot the data in a python script. Figure \ref{fig:kolumbus_15_17} shows the results.

\begin{figure}[ht]
\includegraphics[width=16cm]{kolumbus_total_num_sold_15_17}
\centering
\caption{Monthly passenger travel with Kolumbus}
\label{fig:kolumbus_15_17}
\end{figure}

\chapter{Results}
This chapter describes the subjective view of the results derived from this thesis's program detailed in chapter three and four. Discussion about the results is elaborated upon in the following chapter.



\section{NPRA}
There are three levels of data available: monthly, weekly and hourly. For this reason, the monthly dataset will be disregarded as there are better data available. Weekly data distinctly show the Norwegian holidays described in table \ref{table:jesus}. It is important to take these days into account when deriving information from these data, as otherwise, it would be easy to conclude wrongly. The most dramatic drop is the summer vacation, which luckily is outside the influenza season anyway. Another challenge with holidays and vacations is that the start and duration change yearly, and because of the Gregorian calendar set dates shift one day up the next weekday for the next year. This needs to be taken into consideration, and possibly weeded out or glossed over in order to avoid misinterpretations.

\begin{center}
\begin{table}[!h]
\begin{tabular}{ | m{9em} | m{10cm}| }
 \hline
 \textbf{Vacation/Holiday} & \textbf{When} \\ [0.5ex] 
 \hline
 Summer vacation & About nine weeks from the midth of June to the end of August  \\ 
 \hline 
 Autumn vacation & One week or a long weekend in September or October, usually in week 39, 40 or 41.\\ 
 \hline
 Christmas holiday & Usually two weeks from the end of December to the start of January\\ 
 \hline
 Winter vacation & usually in week 7, 8 or nine in February or March \\ 
  \hline
 Easter holiday & 10-11 days at the end of March or beginning of April \\ 
  \hline
 Other and Christian holy days & Labour Day, Ascension Day, Constitution Day \\ 
  \hline
\end{tabular}
\caption{The Norwegian holidays and vacations}
 \label{table:jesus}
\end{table}
\end{center}

Further, the weekly graphs show a considerable increase in traffic each year, when asked about this the NPRA admitted to their action plan to increase the numbers of traffic registration stations yearly. This increase of infrastructure is transparent in the graphs shown as jumps in the amount of traffic with each new year. From the end of November to the beginning of December there is a slight drop in the amount of traffic without there being any vacations or holidays, this anomaly might be correlated with the influenza season as numbers of reported virus observations and ILI incidents seems to increase at the same time. After influenza, occurrences spike and begin to decline traffic slightly return to normal over the course of January to June.
The weekly data is an aggregated set of many traffic registration stations based on the cities or the all of Norway, therefore roadworks, accidents or closed roads is not directly apparent. They are however visible, by assumption, on the hourly datasets as they only show one traffic registration station at a time. When in doubt of closed roads one could pick another traffic registration station nearby and see if it is also affected in the same manner. Another advantage with the hourly datasets is that there is not a dramatic yearly increase of traffic, which means more reliable data can be obtained, especially from the older traffic registration stations that have been operational for several years already. The map next to the hourly graph shows the available traffic registration stations to choose from.




\section{Twitter}
The way that twitter\_analyser.py works are that it simply counts the number of occurrences of tweets and then draws a graph based on that count. The Twitter data collected in twitter\_data.txt still contains duplicates although efforts were taken to prevent this. The duplicates may affect the graph drawn in batches as spikes where articles or hype are written about influenza or with other of the search terms. The Twitter data has a distinct pulse following the time when people post messages on social media the most by week\cite{socialTrend}, Mondays to Thursdays have a high yield of tweets, and then the weekends are calmer. This at least shows that the data collected is somewhat in accordance with other social media in other parts of the world. During the collection of tweets the event of the Norwegian Easter holiday occurred, from the graph shown the spikes even out and there is a more consistent flow of tweets throughout the holiday.
When comparing the datasets of twitter and NIPH there is a clear similarity between them. The Twitter data seems to follow the trend downwards with the NIPH when the season is coming close to an end. However, the Twitter data seems to be lagging behind by 10 weeks, even less so with the ILI data from Bergen. This is in direct contradiction with both research referenced earlier in chapter two, and with this thesis's expectations.





\section{Kolumbus}
The Kolumbus data is the least interesting as it does not have spatial specific data and that the data resolution is too low on a monthly basis to see any anomalies. The longer Norwegian vacations and holidays are still somewhat visible though. This goes to show that sufficient temporal resolution is critical in order to derive any useful information from data in this thesis.




\section{Ruter}
Comparing the Ruter data with the ILI data of Oslo is especially interesting because Ruter is the public transportation administrator in that city. As with the NPRA data the Norwegian holidays and vacations are apparent as described in section 5.1. The weeks of 47, 48 and 49 show a slight decrease of passenger travel without overlapping any vacations and holidays, there is also a slight increase of reported ILI every influenza season in those weeks. This correlation might be relevant and should be investigated further.
After the Christmas holiday passenger travel struggle for a few weeks to 'catch up' to a more stable level, interestingly enough the influenza seasons usually are on its peaks at that very time.
The amount of passenger travel also seems to be slightly increasing as the influenza season declines.

\chapter{Discussion}
In this chapter, project management and resolutions will be discussed, followed by a short note on ethical concerns considering collecting citizen data, then known bugs and other imperfect implementations will be elaborated upon, lastly possible future works are considered.

\section{Project Management}

Early in the planning and management phase of this thesis, it became evident that the Norwegian infrastructure for retrieving data from various public sources by API was not sufficient for the needs of this thesis. Therefore the initial plan to automate the collection of data needed was adapted to the means of acquiring the data by manually asking the various agencies and implementing their data hard-coded.
\\
This makes the program much less scalable and flexible than hoped for, and severely inhibits future contributions as it may be difficult to couple new data with the inputs of the backend's data structure. The missing automation part will probably hinder future use of this program. The only two APIs used are of American origin, namely Google static map and Twitter. In these regards the automation element that this thesis anticipated failed, however not by a critical means as manual retrieval of data was still possible.






\section{Project resolutions}
In the middle of the time scope for this thesis, a frontend to the backend was desired and thus planning to construct this began. There were several options for choosing not only from the languages the GUI would be based upon but consideration of how a map would be projected as well. These were the main concerns and had to be compatible with each other. Considerable time and effort were spent examining the available choices, and finally, a feasible solution for the purposed implementation was found that could provide the basic needs for the program of this thesis.

\subsection{Graphical User Interface}
The first choice was between Python's Tkinter GUI module and Node/Javascript GUI. The main reason Python was chosen was that it offered the easiest integration with the backend. Javascript prohibits writing or reading from arbitrary files by design, and thus the backend would preferably have to be mounted on a server in order to provide its functions to a frontend. The author of this thesis had little experience with this, and learning a whole new trade was daunting and seemed insurmountable within the rest of the time scope of this thesis, therefore the enticing of the familiarity of Python triumphed. In hindsight, it would probably be better to undertake a Node/Javascript approach because some sort of database to store the backend's data is needed anyway and is probably a more feasible solution. Tkinter is also a framework in which a programmer relinquishes some control for its benefits, this is a minor but notable challenge if it is not familiar.
\\
The advantage the Python solution has is that it requires few installations of external modules and is easily downloaded and mountable on many platforms.
The disadvantage with Python's module Matplotlib, which is used to draw graphs, is that drawing many graphs requires a lot of memory and processor resources, therefore it is important to manage the graphs drawn, and only load those that need be loaded at a time, flushing those that are no longer in use.

\subsection{Google Static Maps}
Choosing a map implementation was difficult, Python has several options like GeoPandas, ipyleaflet, Google static map, cartopy, OpenStreetMap, and basemap. All of the mentioned was hard to install and was sorely limited in function and potential, except OpenStreetMap and Google static map. Upon further investigation Goompy, as described in chapter 2, was discovered and offered a nearly effortless implementation of the map in the already applied design of the frontend.
\\
The advantage of Google static map is that it is a well implemented and established service with consistent qualitative measures. Google offers fewer road details than OpenStreetMap, and that serves this thesis perfectly as the visualization needed was simply showing locations of traffic registration stations and not necessarily other roads. 
\\
The disadvantages with Google static map is that there are standard usage limits (which can simply be overcome with paying for more). Pixel resolution is set to a maximum of 640x640 pixels, and the free usage is limited to 25.000 map loads per 24 hours. These two limits are not really a problem: The pixel limit is overcome by simply requesting more map loads and then combining those to create as big a picture as desired, and the map loads limit is very high. On average Goompy does 4 map loads per zoom (thus creating a big map of 1280x1280 pixels) and 25.000 / 4 = 6250 zooms per 24 hours, average Norwegian working hours per day is 7.5 hours, this means that one would reach the limit if there are 6.250 / 7.5 / 60 = 13.9 zooms per second. 
\\
This limit was never reached in testing and although it is a high limit if reached the map simply stops working for the remainder of the time to the next 24 hours. Perhaps the most severe limit Google static map have for the scope of this thesis is its maximum URL size of 8192 characters. Figure \ref{fig:google_url} show the programs URL that it sends to the Google map servers, containing a standard map and fifty-three traffic registration stations each with their individually different sizes and colors this surmounts to a total of 4.975 characters already, which is 60.7\% of the total allowed. 

\begin{figure}[!htb]
\includegraphics[width=11cm]{Google_static_map_URL}
\centering
\caption{Size of the programs Google static map URL per request.}
\label{fig:google_url}
\end{figure}

Although the url have encoded polylines, which compresses the data, it is already quite long, Loading all of Norway's current 10.066 traffic registration stations using Google static map with this thesis's current algorithms is not feasible, although this could be solved by clustering the traffic registration stations together, and only loading what you actually can see on the map. This would require more Goompy modifications by somehow fetching only those traffic registration stations that are actually currently visible on the map.
\\
The program would still be considered modular and scalable with the chosen technologies and implemented algorithms, although better solutions may be applied.






\section{Ethics}
This potential technology presented is merely a utility information tool that could be used or misused. The question examined is is it acceptable to surveillance the population? And to what extent?
\\
This thesis does not endorse or suggest oppressive mass surveillance or want to have a consequence of altering peoples behaviour. The data used is aggregated beyond personal identification, exempli gratia the NPRA data only shows the number of vehicles passed and in no way can trace individual behaviour. A system for detecting influenza does not need data on an individual level. 
\\
It is considerate to contemplate ethical values when developing systems that rely on public information extraction, taking care to not be reckless with sensitive data by insincere or cynical means, and having a genuine interest in the well-being of the public without consequences of oppression or discrimination. This is the reason the NIPH ILI daily data from Oslo and Bergen are omitted in the delivery of this thesis even though the data is aggregated beyond identification, as the possible information derived might be too delicate, and also possibly protected by Norwegian confidentiality laws. Norway already has a government agency installed protecting such concerns namely the Norwegian Data Protection Authority\cite{datatilsynet}, albeit new technologies and usage are revealed often it is important to have an ongoing update on such policies.
\\

An example that goes to far with collecting and using citizen data would be the Chinese Sesame 'Social credit system' program\cite{meissner2017china}\cite{china_botsman}\cite{china_wiki}, where a propaganda game rating citizens with 'Sesame credits' on their lives and judging them according to their 'trustworthiness' and 'social integrity' of their individual behaviour. Behaviour such as local and online purchases, real-time location, who friends and family are and what they do, the content of leisure and payment of bills. This mass surveillance tool uses big data analysis technology, and in contradiction to official intents acts as an oppressive system punishing its subjects. Examples of punishment are flight bans limiting movement, excluding parent's children from enrolling in private schools, slow internet access, exclusion from certain jobs, exclusion from hotel services, and forced registration on a public blacklist. The Sesame credit system also works for businesses in their own way. The Chinese system depends on individual data and analysis with intent to influence a person's etiquette.






\section{Bugs and other imperfect implementations}
One known negligent solution is the algorithms in the program of double\_y\_graphs.py where redundant calculations take place. Fixing this would make the program gui.py slightly faster and be more structural preferable. The difficulty lies with not reusing already created objects and instead forge a new, this is in probity a crude imposition underneath expected proficiency.\\
The program NIPH\_frame.py serves the function to present two graphs on the same x-axis and with their own separate y-axes, this was however only accomplished on a weekly temporal resolution. Some graphs have a higher resolution like for instance Twitter, but the data is still aggregated into a weekly resolution in the program. Writing algorithms that would support an hourly or weekly resolution became outside of the time scope of this thesis. Future work may focus on being able to compare different resolutions as this would offer a better comparison of the different datasets.\\
Another known bug is that the buttons panel disappear when the window size is not big enough, the attempt to amend this has again and again produced frustration and the answer remains to this day a mystery.
The NPRA hourly dataset GUI implementation is missing the option to choose from different traffic registration stations. One reason this was not prioritised in time was that the data obtained was in an older data structure and conversion was difficult, though the one available hourly data set in the program proves the concept of manipulating and studying the NPRA statistics.\\
Some functions in the different modules both in the backend and in the frontend are redundant, a better overall structure would be to collect these often used functions in their own utility module. An example of this may be the drawing of graphs in the backend. Having a draw module that only draws whatever the different modules require would serve as a uniform utility tool. This was implemented lastly in the frontend with the program constants.py which only serves the shared constants for the color theme. This makes it easy to change because one would only have to alter it in one place, not having the need to search through every place it is implemented in code. A solid utility module for both the backend and the frontend should have been achieved for better structure and optimisation.








\section{Future works}
The program contains bugs and inefficient solutions, ameliorating these algorithms requires more work than what is this thesis's time scope. There is a multitude of different open sources, tools, and frameworks in existence. Choosing the right technology and solution for the right project and problem is a challenging task that perhaps becomes easier with experience and adequate knowledge. The state of ever-changing available technologies makes it so viable options change rapidly, and the adherence to adapt is an ever-evolving developer skill. This thesis could have been better served with additional forethought which would require supplemental research on applied standardisation and feasible solutions to structure and wanted features. Although the presented work is within the desired outcome, better realisation of necessities and accessibilities may have been further advantageous.

\subsection{Test driven development}
Test-driven development (TDD) is the exercise of writing tests for code even before the creation of the algorithm to be tested upon. The goal is to specify the exact parameters and functions an algorithm should have by writing a test firstly, and then writing the actual algorithm and making it pass the test. Although this is a big investment that essentially adds another layer of complexity and requires continuous tweaking the advantages are imposing. A clear acceptance criteria safely define the purpose should one be left astray, and convey a focus on integration, control and well-organized code for safer refactorisation and fewer bugs. The toll of utilizing TDD is high inherently but quickly offers increasing returns and is a virtuous investment that also serves as a living document.
In hindsight, it is the belief that this thesis would benefit greatly from this practice, and should be considered a future contrivance if this thesis should ever be rewritten by others.

\subsection{Database}
At the very end of the time scope of this thesis, it became obvious that the backend's data should have been implemented in some sort of a database in order to speed up the process of reading and extracting exact information, this especially relevant for the NPRA hourly dataset. Data filtering is the process of refining data sets for relevant user information, different filters can be tailored to different needs. Filtering becomes particularly useful in the NPRA's hourly dataset, an example would be to filter out the different vehicle lanes available, this would on average make the algorithm two times faster. In order to take advantage of data filtering and indexing the data would have to be implemented in a database. A possible more optimal solution would be to rewrite the entire project to Node/Javascript and mount the backend's data on a server such that it is available to the frontend modules.

\subsection{Google static map}
As discussed in the previous chapter clustering traffic registration stations would solve the maximum URL problem. When presented with a map that shows all of Norway instead of showing each traffic registration stations one could cluster them together by proximity, and when zooming in present an even more fine-tuned clustering until the zoom level is sufficient enough to show all of the traffic registration stations on that level. This is already a standard way of presenting spatial data as seen in the NPRA's online roadmap\cite{vegkart} when selecting multiple elements. Further standardising colors and sizes would significantly save url length, the thought behind different sizes and colors was only intended on a very zoomed in level and is not necessarily needed when showing clusters. Considering when and what is needed amends the complication. The url size problem would also completely vanish if taken a Node/Javascript approach instead.

\subsection{Additional features}
A wanted feature was the variance calculation of the different traffic registration stations that had hourly dataset on them. This would visualize the data on the map in an interestingly manner by differing the sizes based on amount of traffic and colourise them based on the difference by each station's yearly variance. This planned feature fell of of this thesis's time-scope, and although outlined never reach actualisation. There are however attempted experimental algorithms in the program NPRA\_Traffic\_Stations\_load\_data.py, the main concern was the time-cost of these algorithms as calculating the variance of one hourly dataset over the course of five years meant reading through about 87.630 lines of data which could take nearly a minute to run on modern computers. This is why having an indexed database would be beneficial when running such an algorithm on all of the 53 traffic registration stations accessible in this thesis. Although a neat feature the lack of a indexed database and insufficient time, this was never completed.




\chapter{Conclusion}
\section{TODO}


\subsection{Future works}
svakheter, hvordan gjøre bedre? hva er mitt bidrag?

\appendix
\chapter{A.1 Source Code}
The source code for the program created for this thesis is found by this link: \url{https://github.com/Slideshow776/DATMAS-V18}. This is also the intended way to view the directory, as the three readme .md files contained are best visualised with their markup language, the github page is also considered a part of this thesis's work. \\
There was no way to append the code to the delivery of this thesis, so the code is available only by this means. There are no guarantees that the link will work.

%\printbibliography
\bibliography{references}
\bibliographystyle{ieeetr}
%\usepackage{url}

\end{document}